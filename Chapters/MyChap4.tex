% Chapter Template

\chapter{Synchronising} % Main chapter title

\label{ChapterX} % Change X to a consecutive number; for referencing this chapter elsewhere, use \ref{ChapterX}

%----------------------------------------------------------------------------------------
%	SECTION 1
%----------------------------------------------------------------------------------------

\section{Array Matching}
With a sufficiently accurate subtitle predictor, the next challenge was to match the predictions to the  true array. Sabater initially took the whole prediction array and computed the log loss over different positions of the subtitles array in order to find the minimum which would equate to the best match. However, this approach assumes full access to both arrays: in order to match subtitles in real time, the probability array would be accessed incrementally as new audio is recorded and features extracted. 

\subsection{Fixed Recording Duration}
Initially, it was modelled that audio of fixed duration would be recorded in order to match array (50 frames initially, equivalent to 500ms), whilst noting the time for the algorithm to run, and applying a delay to the subtitles once the match had been found. The length of the audio was to be optimised, and it was assumed that live audio data would commence acquisition after the beginning of the film so that the algorithm does not attempt to match audio features generated before the film has actually begun, perhaps due to background noise or advertisments. To test this independently of the subtitles predictor, the array match was given a sample of the subtitles array N steps after the beginning of the full subtitles array in order to evaluate the behaviour. Interestingly, this approach consistently returned index 0 as the best position. On investigation, it was revealed that since there were no subtitles at the beginning of the film the sample array was made up of entirely zeros, and so the function returned this as the optimal spot: there were in fact multiple optimum positions (generally all at the start) where no subtitles were present and it simply returned the first position which fulfilled the minimisation condition. 

\subsection{Live Recording}
Whilst this seemed like an obstacle initially, it was realised that the duration of time in which no subtitles occurs can be considered a distinctive feature in itself, especially when combined with the assumption that synchronisation has commenced not long after the beginning of the film: it is unlikely that viewing would commence after the majority of the film has already finished. In addition, new data can be continuously acquired by simply appending new predictions based on new features extracted from the incoming audio. This foreknowledge, a product of the specific nature of this problem, can be utilised to shrink the search space significantly by defining the first match to be obtimal, whilst continuously expanding the number of examples to match against the base truth. In practice this is implemented by recording data and extracting the features, and if no speech is found, continue to acquire new data until a subtitle is predicted. This subtitle would correspond to the first subtitle present in the film and would provide the distinctive feature to search for when matching. This could be done in 2 ways, either the array generated, of a long sequence of zeros followed by a 1, is matched using the log loss, or in an even greater simplification, the subtitles are simply commenced as soon as speech is detected. Alternatively, if speech commences from the beginning, the recorded features should provide enough variation that a match can be found when searching from the beginning.